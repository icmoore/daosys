\documentclass[10pt,xcolor=svgnames]{beamer} %Beamer
\usepackage{palatino} %font type
\usepackage{xcolor}
\usefonttheme{metropolis} %Type of slides
\usefonttheme[onlymath]{serif} %font type Mathematical expressions
\usetheme[progressbar=frametitle,titleformat frame=smallcaps,numbering=counter]{metropolis} %This adds a bar at the beginning of each section.
\useoutertheme[subsection=false]{miniframes} %Circles in the top of each frame, showing the slide of each section you are at

\usepackage{appendixnumberbeamer} %enumerate each slide without counting the appendix


\definecolor{SysBlue}{RGB}{12,120,183}
\definecolor{Progress}{RGB}{0,137,208}
\definecolor{Header}{RGB}{12,120,183}
\definecolor{SlideTitle}{RGB}{0,137,208}
\definecolor{ToC}{RGB}{51,58,66}

\setbeamercolor{progress bar}{fg=white} %These are the colours of the progress bar. Notice that the names used are the svgnames
\setbeamercolor{title separator}{fg=SysBlue} %This is the line colour in the title slide
\setbeamercolor{structure}{fg=black} %Colour of the text of structure, numbers, items, blah. Not the big text.
\setbeamercolor{normal text}{fg=black!87} %Colour of normal text
\setbeamercolor{alerted text}{fg=DarkRed!60!Gainsboro} %Color of the alert box
\setbeamercolor{example text}{fg=ToC} %Colour of the Example block text


\setbeamercolor{palette primary}{bg=SlideTitle, fg=white} %These are the colours of the background. Being this the main combination and so one. 
\setbeamercolor{palette secondary}{bg=Header, fg=white}
\setbeamercolor{palette tertiary}{bg=Header, fg=white}
\setbeamercolor{section in toc}{fg=ToC} %Color of the text in the table of contents (toc)

%These next packages are the useful for Physics in general, you can add the extras here. 
\usepackage{amsmath,amssymb}
\usepackage{slashed}
\usepackage{cite}
\usepackage{relsize}
\usepackage{caption}
\usepackage{subcaption}
\usepackage{multicol}
\usepackage{booktabs}
\usepackage[scale=2]{ccicons}
\usepackage{pgfplots}
\usepgfplotslibrary{dateplot}
\usepackage{geometry}
\usepackage{xspace}
\newcommand{\themename}{\textbf{\textsc{bluetemp}\xspace}}%metropolis}}\xspace}

\title{DAOSYS}
\author[Name]{Ian Moore, PhD \inst{$\dagger$} and Cyotee Doge \inst{$\dagger\dagger$}}%With inst, you can change the institution they belong
\subtitle{Smart DAO Protocol for Decentralized Finance}

\institute[shortinst]{\inst{$\dagger$} Syscoin Researcher, Syscoin Platform (e-mail: imoore@syscoin.org) \and %
                      \inst{$\dagger\dagger$} DAO Advisor, Syscoin Platform (e-mail: cyotee@syscoin.org)}

\date{October 27, 2022} %Here you can change the date
\titlegraphic{\vspace{-0.5cm}\hfill\includegraphics[scale=0.23]{logo.png}} %You can modify the location of the logo by changing the command \vspace{}. 

\begin{document}
{
\setbeamercolor{background canvas}{bg=white, fg=black}
\setbeamercolor{normal text}{fg=black}
\maketitle
}%This is the colour of the first slide. bg= background and fg=foreground

\metroset{titleformat frame=smallcaps} %This changes the titles for small caps

\begin{frame}{Outline}
  \setbeamertemplate{section in toc}[sections numbered] %This is numbering the sections
  \tableofcontents[hideallsubsections] %You can comment this line if you want to show the subsections in the table of contents
\end{frame}

\begin{frame}{Objectives}
\underline{\textsc{Some text:}}
\begin{small}
This is some small Text. 
\end{small}

\metroset{block=fill}
\begin{exampleblock}{\textsc{Example block}}
\begin{itemize}
    \item You know how to do itemize
    \item Also here
\end{itemize}
\end{exampleblock}
\end{frame}


\section{Introduction}

\begin{frame}[fragile]{Introduction: blah blah} %You can change fragile by standout
Text Text Text Text. \\You can change the size of the footnote text like  Text\footnote{\small{ here.}} Text\footnote{\large{And here.}} Text\footnote{\tiny{And here.}}
\begin{itemize} %The symbol of the items can be changed by which ever you want, this is just an example.
    \item[$\diamond$] Text,
    \item[$\diamond$] Text,
    \item[$\diamond$] Text.
\end{itemize}
An equation without number could be represented by:
\begin{equation*}
    c^{2} = a^{2} + b^{2}
\end{equation*}
That's all for this slide.
\end{frame}

\begin{frame}[standout]{This is other type of slide}
There is some text here.
And an equation with number:
\begin{equation}
    E^{2} = m^{2} + p^{2}
\end{equation}
\end{frame}

\include{Othersection}

\section{This is another section}
\begin{frame}{Frame Title} %You can also not write fragile or standout and you can see how it looks
    Hello world!
\end{frame}

\section{Final section}

\begin{frame}{Conclusion}
    These are the final words, you do your best to try to wake up everyone that was listening to your talk.
\end{frame}

{\setbeamercolor{palette primary}{fg=black, bg=white} %You can change the colours
\begin{frame}[standout]
  Thank you! And thank to yourself because you did all the job. 
\end{frame}
}
\appendix

\begin{frame}{Back up}
    These slides won't appear in the table of contents and will not be counted as the total slides.
\end{frame}

\end{document}